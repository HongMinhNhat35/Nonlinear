\section{Questions}

\begin{problem}[10]
    Consider the following dynamics of a quadrotor, 
    \begin{align*}
        m \ddot{x} &= - (u_1+u_2) \sin{(\theta)} \\
        m \ddot{y} &= (u_1+u_2)\cos{(\theta)}-mg \\
        I \ddot{\theta} &= r(u_1-u_2)
    \end{align*}
     with control inputs $u_1$ and $u_2$.
     Define the state $q$ and 
     rewrite the dynamic in the form of state-space representation $\dot{q}=f(q,u)$.
    \begin{center}
        \includegraphics[width=0.3\textwidth]{quadrotor.png}
    \end{center}
\end{problem}

\begin{answer}
    The state vector $q$ can be defined as:
	\begin{align*}    
    \mathbf{q} = \begin{bmatrix} x \\ y \\ \theta \\ \dot{x} \\ \dot{y} \\ \dot{\theta} \end{bmatrix}
    \end{align*}
    The state space representation $\dot{q}$ can thus be defined as:
    \begin{align*}   
    \mathbf{\dot{q}} = \begin{bmatrix} \dot{x} \\ \dot{y} \\ \dot{\theta} \\ \ddot{x} \\ \ddot{y} \\ \ddot{\theta} \end{bmatrix}
    \end{align*}
    Based on the given dynamics, we can deduce that:
    \begin{align*}
        \ddot{x} &= {\frac{(- (u_1+u_2) \sin{(\theta)})}{m}} \\
        \ddot{y} &= {\frac{((u_1+u_2)\cos{(\theta)}-mg)}{m}} \\
        \ddot{\theta} &= {\frac{r(u_1-u_2)}{m}}
    \end{align*}
    Therefore,$\dot{q}$ can be defined as:
    \begin{align*}   
    \mathbf{\dot{q}} = \begin{bmatrix} \dot{x} \\ \dot{y} \\ \dot{\theta} \\ {\frac{(- (u_1+u_2) \sin{(\theta)})}{m}} \\ {\frac{((u_1+u_2)\cos{(\theta)}-mg)}{m}} \\ {\frac{r(u_1-u_2)}{m}} \end{bmatrix}
    \end{align*}
\end{answer}

\begin{problem}[10]
    Consider the nonlinear system
    \begin{align*}
        \dot{x}_1 &= x_2 \\
        \dot{x}_2 & = -a \sin{(x_1)} - bx_2
    \end{align*}
    with $a,b>0$.
    Find all the equilibrium points of this system.
\end{problem}

\begin{answer}
    To find equilibrium points of the nonlinear system, $\dot{x_1}$ and $\dot{x_1}$ equals to 0.
    So:
    \begin{align*}
        \dot{x}_1 &= x_2 = 0 \\
        \dot{x}_2 & = -a \sin{(x_1)} - bx_2 = 0
    \end{align*}
    So $x_2=0$ => $\sin{(x_1)} = 0$ => $x_1=n \pi$ where $n = 1,2,...$
\end{answer}

\begin{problem}[15]
    Let $X$ be a compact subset of $\mathbb{R}^n$. Prove that a contraction mapping $P: X \rightarrow X$ is a uniformly continuous function.
\end{problem}

\begin{answer}
    P is contraction mapping on X, therefore:
    \begin{align*}
    		d(P(x),P(y))<kd(x,y)(k \in [0,1))
    \end{align*}
    There exists a $\epsilon$ such that:
    \begin{align*}
    		d(x,y)<\epsilon
    \end{align*}
    There exists a $\delta$ such that:
    \begin{align*}
    		\delta &= \frac{\epsilon}{k}
    \end{align*}
    As such, we have:
    \begin{align*}
    		d(P(x),P(y))<\delta
    		d(x,y)<\epsilon
    \end{align*}
    Therefore, P is a uniformly continuous function.
\end{answer}

\begin{problem}[15]
    Consider the map
    \begin{align*}
        T(x) = rx(1-x)
    \end{align*}
    where $x \in \mathbb{R}$ (this is known as the "logistic" map). Define a set $U=[0,1]$. For what value of $r$ is this function $T$ a contraction map from $U$ to $U$?
\end{problem}

\begin{answer}
    To find r, we have to find derivative of T(x)
    \begin{align*}
    		T'(x)=r(1-2x)    		
    \end{align*}
    Since $U= [0,1]$:
    \begin{align*}
    		T'(0)=r
    		T'(-1)=-r
    \end{align*}
    Since T' is a linear function, max|T'(x)| = r
    For T to be a contraction map:
    \begin{align*}
    max|T'(x)| = r<1
    \end{align*}
    Therefore, r<1 so that T is a contraction map.
\end{answer}

\begin{problem}[15]
    Show that if $f_1: R \rightarrow R$ and $f_2: R \rightarrow R$ are locally Lipschitz, then $f_1+f_2$, $f_1f_2$ and $f_1 \circ f_2$ are locally Lipschitz. $\circ$ denotes function composition.
\end{problem}

\begin{answer}
    If  $f_1$ is locally Lipschitz, then:
    \begin{align*}
        |f_1(x_1)-f_1(x_2)|\leq k_1|x_1-x_2| (1)
    \end{align*}
    If  $f_2$ is locally Lipschitz, then:
    \begin{align*}
        |f_2(x_1)-f_2(x_2)|\leq k_2|x_1-x_2|(2)
    \end{align*}
    Then with (1) and (2), we have:
    \begin{align*}
    		f_1+f_2\\
        =>|[f_1(x_1)+f_2(x_1)]-[f_1(x_2)+f_2(x_2)]| \\
        \leq|f_1(x_1)-f_1(x_2)|+|f_2(x_1)-f_2(x_2)|\\
        \leq (k_1+k_2)|x_1-x_2|
    \end{align*}
    So $f_1+f_2$ is locally Lipschitz
    \begin{align*}
    		f_1f_2\\
        =>|f_1(x_1)f_2(x_1)-f_1(x_2)f_2(x_2)]| \\
        = |f_1(x_1)f_2(x_1)-f_1(x_1)f_2(x_2)+f_1(x_1)f_2(x_2)-f_1(x_2)f_2(x_2) |\\
        =|f_1(x_1)(f_2(x_1)-f_2(x_2))+f_2(x_2)(f_1(x_1)-f_1(x_2))|\\ 
        \leq |f_1(x_1)k_1+f_2(x_2)k_2||x_1-x_2|
    \end{align*}
    Let $L_1=sup_{x \in B(x_0,r)}|f_1(x_1)| $ and $L_2=sup_{x \in B(x_0,r)}|f_2(x_2)| $. The inequality above turns to:
    \begin{align*}
    		|f_1(x_1)(f_2(x_1)-f_2(x_2))+f_2(x_2)(f_1(x_1)-f_1(x_2))|\\ 
        \leq |L_1k_1+L_2k_2||x_1-x_2|
    \end{align*}
    So $f_1f_2$ is locally Lipschitz
    \begin{align*}
    f_1 \circ f_2\\
    => |f_1(f_2(x_1)) - f_1(f_2(x_2))| \leq k_1|f_2(x_1)-f_2(x_2)|\\
    \leq k_1k_2|x_1-x_2|
    \end{align*}
    So $f_1 \circ f_2$ is locally Lipschitz
\end{answer}

\begin{problem}[15] \cite{khalil2002nonlinear}
    Derive the sensitivity equations (in Python/MATLAB) with symbolic computation for the system
    \begin{align*}
        \dot{x}_1 &= \tan^{-1}(ax_1) -x_1x_2 \\
        \dot{x}_2 &= bx_1^2 -cx_2
    \end{align*}
    as the parameters $a,b,c$ vary from their nominal values $a_0=1, b_0=0$, and $c_0=1$.
\end{problem}

\begin{answer}
    \begin{lstlisting}
    % Define symbolic variables
syms x1 x2 a b c

% Define the system of equations
f1 = atan(a * x1) - x1 * x2;
f2 = b * x1^2 - c * x2;
M = [f1,f2];
lambda = [a,b,c];
x = [x1,x2];
A = jacobian([f1;f2],x);
B = jacobian([f1;f2],lambda)

    \end{lstlisting}
    We can denote $\lambda_0 =(a_0,b_0,c_0)=(1,0,1)$. Therefore, the sensitivity equation can be derived as:
    \begin{align*}
    		dot{S(t)}=[\begin{matrix}
    		a/(x_1^2 + 1) - x_2 & -x_1\\
    		0&-1
    		\end{matrix}]S(t)+
    		[\begin{matrix}
    		\frac{x_1}{x_1^2+1} & 0 & 0 \\
    		0 & x_1^2 & x_2^2
    		\end{matrix}] \\
    		S(t_0) = 0
    \end{align*}
\end{answer}

\begin{problem}[20]
    Consider the ordinary differential equation (ODE)
    \begin{align*}
        \frac{dx}{dt}=f(t,x), x(t_0) = x_0,
    \end{align*}
    where $f:\mathbb{R}\times \mathbb{R}^n\rightarrow\mathbb{R}^n$ is bounded and satisfies global Lipschitz condition on an interval $[t_0,t_1]$. Recall that the solution of this ODE can be written in the form as
    \begin{align*}
        x(t) = x_0 + \int_{t_0}^{t} f(s,x(s)) ds,\, \forall t\in[t_0,t_1]
    \end{align*}
    which is essentially the unique fixed-point of the contraction mapping $P[x](t) = x_0 + \int_{t_0}^{t} f(s,x(s)) ds$ defined on the Banach space $C^0([t_0,t_1])$. We also know that this fixed-point can be obtained iteratively using $x_{n+1} = P[x_{n}]$ and letting $n\rightarrow \infty$ (that is, $x^* = \lim_{n\rightarrow\infty} x_n$.)
    
    Let f(t,x)=7tx with x(3)=5, and the solution to this ODE is $x(t)=5 e^{\frac{7}{2}(t^2-9)}$. Verify this solution by applying this contraction mapping iteratively and demonstrate its convergence (in Python/MATLAB) with a plot that showing the approximation error relative to the analytical solution gradually decreases.

    HINT: $\sum_{k=0}^{\infty} \frac{x^k}{k!} = e^x$
    \end{problem}
\begin{answer}
    \begin{lstlisting}
t0 = 3;
t1 = 5;
x0 = 5; 
N = 100; 
max_iter = 10; 
t = linspace(t0, t1, N);
x_exact = @(t) x0 * exp((7/2) * (t.^2 - t0^2));
x_prev = x0 * ones(1, N);
errors = zeros(max_iter, 1);
for iter = 1:max_iter
    x_next = x0 + arrayfun(@(tt) integral(@(s) 7.*s.*interp1(t, x_prev, s, 'linear'), t0, tt), t);
    error = abs(x_next - x_exact(t));
    errors(iter) = max(error); % Store the maximum error
    x_prev = x_next;
end

figure;
plot(1:max_iter, errors, '-o');
xlabel('Iteration Number');
ylabel('Maximum Approximation Error');
title('Convergence of Fixed-Point Iteration');
grid on;
    \end{lstlisting}
    \begin{center}
        \includegraphics[width=0.9\textwidth]{esterr.png}
    \end{center}
\end{answer}

