\section{Report guidelines}

For this course, your homework will usually have three parts: theory, programming, and writing. The design of each assignment aims to help you get a better grasp of what you've learned in class and how to apply it in your own problems.
You are expected to break down complex problems into a sub-problems to solve them. 

We provide this \LaTeX \ template to make it easier for you to submit your solutions. Aalto is providing \href{https://www.aalto.fi/en/services/aalto-university-on-overleaf}{free Overleaf Professional accounts} for all students. We recommend you upload this template to overleaf.com if you don't have latex installed. Please check \href{https://tug.ctan.org/info/short-math-guide/short-math-guide.pdf}{this link} for short math guide for latex.
Additionally, We have a minimum standard for the format of your submissions and strongly recommend that you verify each report against the following criteria before submission:

\begin{itemize}
    \item Unless specified otherwise, reports should be a minimum of 2 pages and a maximum of 5 pages in length.
    \item Include title and author information.
    \item Ensure that solutions are understandable without the need to refer to external sources.
    \item Provide detailed yet clear and concise solutions.
    \item Solutions should be reproducible based on the information provided in the report.
    \item Maintain a consistent structure and language that is easy to follow.
    \item Every figure and table must be referred to in the text and have a caption.
\end{itemize}

Although the primary focus of this course is not on programming, we uphold basic standards for clarity and conciseness in your code:

\begin{itemize}
    \item Your code should be able to reproduce the results you wrote in your report.
    \item Each function, including the main one, should be short and to the point. Ideally, the length of each function should be about 25 lines, and no more than 35. (max 120 characters each line.)
    \item Put comments on each function to explain what it does. Make sure the function names make sense and match the comments. Add comments where needed in the main function.
\end{itemize}

For the final group project, submit a signed confirmation letter of each member's contributions and their weight and show it as the second slide of the final presentation. Your personal score of the final project will be the group score multiplied by your confirmed contribution weight.

Finally, \textcolor{red}{\textbf{remember that plagiarism is never tolerated at Aalto.}} However, collaboration among your peers is welcome. Always follow good academic practices when writing your assignments. \href{https://www.aalto.fi/en/services/turnitin-an-originality-checking-and-feedback-software}{Turnitin} is there to help you check your work for originality.